\documentclass[10pt,a4paper,draft]{report}
\usepackage{polski}
\usepackage[utf8]{inputenc}
\usepackage{amsmath}
\usepackage{amsfonts}
\usepackage{amssymb}
\usepackage{dsfont}
\usepackage{enumerate}
\usepackage{aliascnt}
\usepackage{hyperref}
\usepackage{cleveref}


\begin{document}

\newtheorem{theorem}{Twierdzenie}
\crefname{theorem}{theorem}{Theorems}
\Crefname{Theorem}{Theorem}{Theorems}


\newaliascnt{lemma}{theorem}
\newtheorem{lemma}[lemma]{Lemma}
\aliascntresetthe{lemma}
\crefname{lemma}{lemma}{lemmas}
\Crefname{Lemma}{Lemma}{Lemmas}

\newaliascnt{corollary}{theorem}
\newtheorem{corollary}[corollary]{Wniosek}
\aliascntresetthe{corollary}
\crefname{corollary}{corollary}{corollaries}
\Crefname{Corollary}{Corollary}{Corollaries}

\newaliascnt{proposition}{theorem}
\newtheorem{proposition}[proposition]{Proposition}
\aliascntresetthe{proposition}
\crefname{proposition}{proposition}{propositions}
\Crefname{Proposition}{Proposition}{Propositions}

\newaliascnt{definition}{theorem}
\newtheorem{definition}[definition]{Definicja}
\aliascntresetthe{definition}
\crefname{definition}{definition}{definitions}
\Crefname{Definition}{Definition}{Definitions}

\newaliascnt{remark}{theorem}
\newtheorem{remark}[remark]{Remark}
\aliascntresetthe{remark}
\crefname{remark}{remark}{remarks}
\Crefname{Remark}{Remark}{Remarks}


\newtheorem{example}[theorem]{Example}
\crefname{example}{example}{examples}
\Crefname{Example}{Example}{Examples}

\newtheorem{problem}{Zadanie}
\crefname{problem}{problem}{problems}
\Crefname{Problem}{Problem}{Problems}

\newenvironment{proof}{\textbf{Dowód}}

\title{Algorytmy optymalizacji w statystyce}


\chapter{Wstęp do optymalizacji wypukłej}


\section{Zbiory i funkcje wypukłe}

\begin{definition}
Podzbiór $W$ przestrzeni liniowej $L$ nad ciałem liczb rzeczywistych $\mathbb{R}$ nazywamy wypukłym wtedy i tylko wtedy, gdy dla dowolnych $x,y \in W$ oraz $\lambda \in [0,1]$ zachodzi $\lambda x + (1 - \lambda)y \in W$. 
\end{definition}

Dowody poniższych dwóch twierdzeń są niemal dokładną kopią ze skryptu do Optymalizacji 2 prof. Jana Palczewskiego \url{http://mst.mimuw.edu.pl/lecture.php?lecture=op2&part=Ch3}

\begin{theorem}[Silne twierdzenie o oddzielaniu]
Niech $U,V \subset \mathbb{R}^n$ będą niepustymi zbiorami wypukłymi, $U$ zwarty oraz $U \cap V = \emptyset$. Wówczas istnieje hiperpłaszczyzna ściśle oddzielająca $U$ i $V$, to znaczy takie $a \in \mathbb{R}^n$, że dla wszystkich $x \in U, y \in V$ zachodzi:
\[
a^T x < a^T y
\]
\end{theorem}
\begin{proof}
Określamy funkcję $d: U \times V \rightarrow \mathbb{R}$ wzorem $d(x,y) = ||x-y||$ dla $x \in U, y \in V$. Funkcja ta na zbiorze $U \times V$ jest ciągła, dodatnia i przyjmuje w pewnym punkcie $(x_0, y_0)$ dodatnie minimum (dowód na ćwiczeniach). Pokażemy że  $a = y_0 - x_0$ spełnia tezę zadania. Wykażemy najpierw, że $a^T y \geq a^T y_0$ dla $y \in V$. Określamy funkcję $g: [0,1] \rightarrow \mathbb{R}$
\[
g(t) = d(x_0, y_0 + t(y-y_0))^2
\]
czyli:
\[
g(t) = ||y_0 - x_0||_2^2 + 2t (y_0 - x_0)^T (y - y_0) + t^2 ||y-y_0||_2^2
\]
Z wypukłości $V$ wynika że dla $t \in [0,1]$ zachodzi $y_0 + t(y - y_0) \in V$, a zatem $g(t) \geq g(0) = d(x_0, y_0)^2$ na mocy definicji $(x_0, y_0)$. Zatem 
\[
g'(0) = \lim_{t \rightarrow 0} \frac{g(t) - g(0)}{t} \geq 0
\]
czyli $2(y_0 - x_0)^T (y - y_0) \geq 0$, co jest równoważne $a^T y \geq a^T y_0$. Analogicznie dowodzimy $a^T x \leq a^T x_0$ dla $x \in U$. Pozostaje nam sprawdzić, że $a^T x_0 < a^T y_0$, co jest jednak równoważne $||a||_2^ > 0$ a zatem jest prawdą.
\end{proof}

\begin{theorem}[Słabe twierdzenie o oddzielaniu]
Niech $U,V \subset \mathbb{R}^n$ będą niepustymi zbiorami  wypukłymi takimi że $U \cap V = \emptyset$. Wówczas istnieje płaszczyzna oddzielająca $U$ i $V$, to znaczy istnieją $a \in \mathbb{R}^n$ takie że dla wszystkich $x \in U, y \in V$ zachodzi
\[
a^T x \leq a^T y
\]
\end{theorem}
\begin{proof}
Niech $C = V - U = \{ y - x: x \in U, y \in V \}$. Zbiór $C$ jest wypukły, i $0 \not\in C$. Rownoważne tezie jest znalezienie takiego $a \in \mathbb{R}^n$ że $a^T x \geq 0$ dla wszystkich $x \ in C$. Definiujemy zbiory:
\[
A_x = \{a \in \mathbb{R}^n : ||a||_2 = 1 , a^T x \geq 0 \}
\]
Wystarczy pokazać, że $\bigcap_{x \in C} A_x \neq \emptyset$
\end{proof}

Przedstawimy jeszcze bardzo użyteczny wniosek ze słabego twierdzenia o oddzielaniu, który też jest często nazywany słabym twierdzeniem o oddzielaniu.
\begin{corollary}

\end{corollary}

\begin{definition} [funkcji (ściśle) wypukłej]
Niech $W \subset L$ będzie wypukłym podzbiorem przestrzeni liniowej $L$ nad ciałem $\mathbb{R}$. Funkcję $f : W \rightarrow \mathbb{R}$ nazwiemy wypukłą, wtw. gdy dla wszystkich $x,y \in W$ oraz $\lambda \in [0,1]$ zachodzi
\[
f(\lambda x + (1-\lambda)y ) \leq \lambda f(x) + (1 - \lambda) f(y)
\]
jeśli nierówność jest ostra dla wszystkich $x \neq y$ oraz $t \in (0,1)$, funkcję nazwyamy ściśle wypukłą.
\end{definition}

\begin{definition} [funkcji silnie wypukłej]
\[
f(tx + (1-t)y) + mt(1-t)||x-y||_2^2 \leq tf(x) + (1-t)f(y) 
\]
\end{definition}

\begin{definition}[Epigraf]
Dla danej funkcji wypukłej $f : W \rightarrow \mathbb{R}$ epigrafem $f$ nazwiemy zbiór
\[
epi (f) = \{ (x,z) : z \geq f(x) \}
\]
\end{definition}

Silna wypukłość jest ważną własnością. Wiele omawianych przez nas algorytmów optymalizacji funkcji wypukłych będzie zbiegać istotnie szybciej dla funkcji silnie wypukłych. Na ćwiczeniach omówimy różne przydatne własności funkcji silnie wypukłych.

\begin{theorem}
Funkcja $f$ jest wypukła wtedy i tylko wtedy gdy jej epigraf jest wypukły.
\end{theorem}
\begin{proof}
Ćwiczenie.
\end{proof}


\section{Subgradient i subróżniczka}

\begin{definition}[Subgradient]
Dla danej funkcji wypukłej $f : W \rightarrow \mathbb{R}$ oraz $x \in W$ subgradientem $f$ w $x$ nazwiemy dowolny wektor $v$, taki że dla wszystkich $z \in W$:
\[
f(z) \geq f(x) + \langle v , z - x \rangle
\]
\end{definition}

\begin{definition}[Subróżniczka]
Subróżniczką funkcji wypukłej $f: W \rightarrow \mathbb{R}$ w $x \in W$ nazwiemy zbiór wszystkich subgradientów $f$ w $x$. Subróżniczkę $f$ w $x$ oznaczamy przez $\partial f(x)$. Formalnie:
\[
\partial f(x) = \{ v : \forall_{z \in W} f(z) \geq f(x) + \langle v, z - x\rangle \}
\]
\end{definition}

\section{Stożki normalne i styczne}

\begin{definition}[Stożka] Stożkiem w dowolnej przestrzeni liniowej nad $\mathbb{R}$ nazywamy dowolny zbiór $C$, taki że dla dowolnego $\lambda \geq 0$ zachodzi $x \in C \leftrightarrow \lambda x \in C$.
\end{definition}

\begin{definition}[Wektora stycznego do zbioru]
Dla danego zbioru $S$ oraz $x \in S$, powiemy że wektor $v$ jest wektorem stycznym do $S$ w $x$, wtedy i tylko wtedy gdy  istnieje ciąg punktów $x_n \in S$, $\lim_{n \rightarrow \infty} x_n = x$, oraz ciag $\lambda_n \in \mathbb{R}_+$, taki że:
\[
\lim_{n \rightarrow \infty} \frac{x_n  - x}{\lambda_n} = v 
\]
\end{definition}

\begin{theorem}
Zbiór wszystkich wektorów stycznych do zbioru $W$ w $x$ jest domknięty oraz tworzy stożek.
\end{theorem}


\begin{definition}[Stożka stycznego]
Stożkiem wektorów stycznych lub stożkiem stycznym do $W$ w $x$ nazywamy zbiór wektorów stycznych do $W$ w $x$. Często będziemy oznaczać ten zbiór przez $T_W(x)$.
\end{definition}

\begin{theorem}
Stożek styczny do zbioru wypukłego jest zbiorem wypukłym.
\end{theorem}
\begin{proof}
Ćwiczenie.
\end{proof}

\begin{theorem}
\end{theorem}

\section{Rzutowanie i operator proksymalny}

\begin{definition}
Niech $W$ będzie domkniętym podzbiorem wypukłym rzeczywistej przestrzeni Hilberta $H$. Funkcję:
\[
P_W(x) = argmin_{y \in W} ||x - y||_H
\]
\end{definition}
\begin{theorem}
Operator rzutowania jest dobrze zdefiniowany.
\end{theorem}
\begin{proof}
Ćwiczenie
\end{proof}


Operator proksymlany jest pewnym uogólnieniem 


\section{Zadania}
\begin{problem}
Udowodnij, że zbiór $W$ jest wypukły, wtedy i tylko wtedy gdy dla dowolnych $x_1, x_2, \ldots x_n \in W$ oraz $\lambda_1, \lambda_2, \ldots \lambda_n \in \mathbb{R}_+$, $\sum_{i=1}^n \lambda_i = 1$, zachodzi
\[
\sum_{i=1}^n \lambda_i x_i
\]
\end{problem}
\begin{problem}
Niech $W \in \mathbb{R}^n$ będzie zbiorem wypukłym o niepustym wnętrzu. Wówczas:
\begin{enumerate}[a)]

\item Dla dowolnego $x \in W$ oraz $x_0 \in int(W)$ odcinek łączący $x$ i $x_0$ z pominięciem jednego z punktów początkowych należy do wnętrza $W$. Formalnie, dla $\lambda \in (0, 1]$ zachodzi
\[
\lambda x + (1 - \lambda) x_0 \in int(W)
\]

\item $W \in cl(int(W))$
\end{enumerate}
\end{problem}
\begin{problem}
Udowodnij, że następujące stwierdzenia są równoważne:
\begin{enumerate}[a)]
\item Funkcja $f: W \rightarrow \mathbb{R}$ jest $\mu$-silnie wypukła
\item Dla dowolnego $z$, funkcja $g(x) = f(x) - \frac{\mu}{2}||x-z||_2^2$ jest wypukła.
\item Dla dowolnego $s_x \in \partial f(x)$ oraz $y \in W$ zachodzi
\[
f(x) \geq f(y) + s_x^T(x - y) + \frac{\mu}{2} ||x-y||_2^2
\]
\item
\end{enumerate}

\end{problem}

\begin{problem}
Znaleźć stożek normalny do wielościanu (przecięcia skończonej liczby półprzestrzeni w $\mathbb{R}^n$). 
\end{problem}
\begin{problem}
Udowodnij że operator rzutowania jest dobrze zdefiniowany
\end{problem}
\begin{problem}
Udowodnij, że funkcja jest wypukła wtedy i tylko wtedy gdy jej epigraf jest wypukły.
\end{problem}



\chapter{Metody pierwszego rzędu}
\section{Spadek gradientowy}
Lemat
analiza dla wypukłych
analiza dla silnie wypukłych
\section{Operator proksymalny}
definicja
proximal iteration

\section{Proksymalny spadek gradientowy}
algorytm
lemat
analiza
\section{Metoda subgradientowa}
algorytm
analiza

\section{Mirror descent}

Sebastian Bubeck, "Five miracles of mirror descent" na YT. Dziewięć wykładów o długości 1h każdy. Nie do końca o optymalizacji, ale dużo ciekawych informacji o MD.

\section{Metody stochastyczne}

\chapter{Przyśpieszenie Nesterova}
Omówimy w tym rozdziale przyśpieszenie (akcelerację) Nestorva. Jest to modyfikacja spadku gradientowego, o szybszej zbieżności do wartości optymalnej.
\section{Algorytm i analiza}

\section{Restarty}


\chapter{Teoria dualności}
\section{Sformułowanie}
Omówimy teorię dualności dla ogólnych problemów optymalizacyjnych:
\[
\begin{array}{c}
min f(x) \\
g_i(x) \leq 0 \\
h_i(x) = 0
\end{array}
\]
Lagrangian \\
Punkt siodłowy \\
Postać dla programów liniowych
\section{Przykłady użycia}
Ze slajdów Tibshiraniego

\chapter{Metody pierwotno-dualne}

\chapter{ADMM}


\end{document}