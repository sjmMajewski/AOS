\documentclass[10pt,a4paper,draft]{report}
\usepackage{polski}
\usepackage[utf8]{inputenc}
\usepackage{amsmath}
\usepackage{amsfonts}
\usepackage{amssymb}
\usepackage{dsfont}
\usepackage{enumerate}
\usepackage{aliascnt}
\usepackage{hyperref}
\usepackage{cleveref}


\begin{document}

\newcommand{\rset}{\mathbb{R}}
\newcommand{\steps}{\gamma}

\newtheorem{theorem}{Twierdzenie}
\crefname{theorem}{theorem}{Theorems}
\Crefname{Theorem}{Theorem}{Theorems}


\newaliascnt{lemma}{theorem}
\newtheorem{lemma}[lemma]{Lemma}
\aliascntresetthe{lemma}
\crefname{lemma}{lemma}{lemmas}
\Crefname{Lemma}{Lemma}{Lemmas}

\newaliascnt{corollary}{theorem}
\newtheorem{corollary}[corollary]{Wniosek}
\aliascntresetthe{corollary}
\crefname{corollary}{corollary}{corollaries}
\Crefname{Corollary}{Corollary}{Corollaries}

\newaliascnt{proposition}{theorem}
\newtheorem{proposition}[proposition]{Proposition}
\aliascntresetthe{proposition}
\crefname{proposition}{proposition}{propositions}
\Crefname{Proposition}{Proposition}{Propositions}

\newaliascnt{definition}{theorem}
\newtheorem{definition}[definition]{Definicja}
\aliascntresetthe{definition}
\crefname{definition}{definition}{definitions}
\Crefname{Definition}{Definition}{Definitions}

\newaliascnt{remark}{theorem}
\newtheorem{remark}[remark]{Remark}
\aliascntresetthe{remark}
\crefname{remark}{remark}{remarks}
\Crefname{Remark}{Remark}{Remarks}


\newtheorem{example}[theorem]{Example}
\crefname{example}{example}{examples}
\Crefname{Example}{Example}{Examples}

\newtheorem{problem}{Zadanie}
\crefname{problem}{problem}{problems}
\Crefname{Problem}{Problem}{Problems}

\newenvironment{proof}{\textbf{Dowód}}

\title{Algorytmy optymalizacji w statystyce}


\chapter{Wstęp do optymalizacji wypukłej}


\section{Zbiory i funkcje wypukłe}

\begin{definition}
Podzbiór $W$ przestrzeni liniowej $L$ nad ciałem liczb rzeczywistych $\mathbb{R}$ nazywamy wypukłym wtedy i tylko wtedy, gdy dla dowolnych $x,y \in W$ oraz $\lambda \in [0,1]$ zachodzi $\lambda x + (1 - \lambda)y \in W$. 
\end{definition}

Dowody poniższych dwóch twierdzeń są niemal dokładną kopią ze skryptu do Optymalizacji 2 prof. Jana Palczewskiego \url{http://mst.mimuw.edu.pl/lecture.php?lecture=op2&part=Ch3}

\begin{theorem}[Silne twierdzenie o oddzielaniu]
Niech $U,V \subset \mathbb{R}^n$ będą niepustymi zbiorami wypukłymi, $U$ zwarty oraz $U \cap V = \emptyset$. Wówczas istnieje hiperpłaszczyzna ściśle oddzielająca $U$ i $V$, to znaczy takie $a \in \mathbb{R}^n$, że dla wszystkich $x \in U, y \in V$ zachodzi:
\[
a^T x < a^T y
\]
\end{theorem}
\begin{proof}
Określamy funkcję $d: U \times V \rightarrow \mathbb{R}$ wzorem $d(x,y) = ||x-y||$ dla $x \in U, y \in V$. Funkcja ta na zbiorze $U \times V$ jest ciągła, dodatnia i przyjmuje w pewnym punkcie $(x_0, y_0)$ dodatnie minimum (dowód na ćwiczeniach). Pokażemy że  $a = y_0 - x_0$ spełnia tezę zadania. Wykażemy najpierw, że $a^T y \geq a^T y_0$ dla $y \in V$. Określamy funkcję $g: [0,1] \rightarrow \mathbb{R}$
\[
g(t) = d(x_0, y_0 + t(y-y_0))^2
\]
czyli:
\[
g(t) = ||y_0 - x_0||_2^2 + 2t (y_0 - x_0)^T (y - y_0) + t^2 ||y-y_0||_2^2
\]
Z wypukłości $V$ wynika że dla $t \in [0,1]$ zachodzi $y_0 + t(y - y_0) \in V$, a zatem $g(t) \geq g(0) = d(x_0, y_0)^2$ na mocy definicji $(x_0, y_0)$. Zatem 
\[
g'(0) = \lim_{t \rightarrow 0} \frac{g(t) - g(0)}{t} \geq 0
\]
czyli $2(y_0 - x_0)^T (y - y_0) \geq 0$, co jest równoważne $a^T y \geq a^T y_0$. Analogicznie dowodzimy $a^T x \leq a^T x_0$ dla $x \in U$. Pozostaje nam sprawdzić, że $a^T x_0 < a^T y_0$, co jest jednak równoważne $||a||_2^ > 0$ a zatem jest prawdą.
\end{proof}

\begin{theorem}[Słabe twierdzenie o oddzielaniu]
Niech $U,V \subset \mathbb{R}^n$ będą niepustymi zbiorami  wypukłymi takimi że $U \cap V = \emptyset$. Wówczas istnieje płaszczyzna oddzielająca $U$ i $V$, to znaczy istnieją $a \in \mathbb{R}^n$ takie że dla wszystkich $x \in U, y \in V$ zachodzi
\[
a^T x \leq a^T y
\]
\end{theorem}
\begin{proof}
Niech $C = V - U = \{ y - x: x \in U, y \in V \}$. Zbiór $C$ jest wypukły, i $0 \not\in C$. Rownoważne tezie jest znalezienie takiego $a \in \mathbb{R}^n$ że $a^T x \geq 0$ dla wszystkich $x \in C$. Definiujemy zbiory:
\[
\begin{aligned}
A_x &= \{a \in \mathbb{R}^n : ||a||_2 = 1 , a^T x \geq 0 \} \\
B_x &= S_{n-1} \setminus A_x 
\end{aligned}
\]
Wystarczy pokazać, że $\bigcap_{x \in C} A_x \neq \emptyset$. Wykażemy to przez sprzeczność. Załóżmy że $\bigcap_{x \in C} A_x = \emptyset$. Wówczas $\bigcup_{x \in C} B_x = S$, zatem $\{B_x\}_{x \in C}$ jest pokryciem otwartym sfery jednostkowej $S_{n-1}$. Ponieważ sfera jest zwarta, istnieje podpokrycie skończone, czyli taki zbiór $\{x_i\}_{i=1}^k$, $x_i \in C$, że $\bigcup_{i=1}^k B_{x_i} = S$. Stąd $\bigcap_{i=1}^k A_{x_i} = \emptyset$. Niech $\hat{C}$ będzie uwypukleniem zbioru punktów $x_i$, czyli:
\[
\hat{C} = \left\{ \sum_{i=1}^k \lambda_i x_i : \lambda_i \geq 0, \sum_{i=1}^k \lambda_i = 1\right\} 
\] 
Zbiór $\hat{C}$ jest oczywiście wypukly, domkniety, oraz $\hat{C} \subseteq C$. Stąd $0 \not\in \hat{C}$, więc z silnego twierdzenia o oddzielaniu dla zbiorów $\{0\}, \hat{C}$ wnioskujemy że istnieje $a \in \mathbb{R}^n$, takie że $a^T x_i > 0$. Jednak wtedy $a / ||a||_2 \in A_{x_i}$ dla wszystkich $\{x_i\}_{i=1}^k$, co daje nam sprzeczność.
\end{proof}

Przedstawimy jeszcze bardzo użyteczny wniosek ze słabego twierdzenia o oddzielaniu, który też jest często nazywany słabym twierdzeniem o oddzielaniu.
\begin{corollary} \label{cor:weak_sep}
Niech $U,V \subseteq \mathbb{R}^n$ będą zbiorami wypukłymi takimi że $int(U) \neq \emptyset$ oraz $int(U) \cap V = \emptyset$. Wówczas istnieje hiperpłaszczyzna rozdzielająca $U$ i $V$, to znaczy takie $a \in \mathbb{R}^n$, że dla wszystkich $x \in U, y \in V$ zachodzi:
\[
a^T x \leq a^T y
\]
\end{corollary}

\begin{definition} [funkcji (ściśle) wypukłej]
Niech $W \subset L$ będzie wypukłym podzbiorem przestrzeni liniowej $L$ nad ciałem $\mathbb{R}$. Funkcję $f : W \rightarrow \mathbb{R}$ nazwiemy wypukłą, wtw. gdy dla wszystkich $x,y \in W$ oraz $\lambda \in [0,1]$ zachodzi
\[
f(\lambda x + (1-\lambda)y ) \leq \lambda f(x) + (1 - \lambda) f(y)
\]
jeśli nierówność jest ostra dla wszystkich $x \neq y$ oraz $t \in (0,1)$, funkcję nazwyamy ściśle wypukłą.
\end{definition}

\begin{theorem}
Niech $W$ będzie zbiorem o niepustym wnętrzu a $f : W \rightarrow \mathbb{R}$ niech będzie funkcją wypukłą. Wówczas $f$ jest ciągła we wnętrzu zbioru $W$.
\end{theorem}
Dowód powyższego twierdzenia wykracza poza ramy tego wykładu. Zainteresowanego czytelnika odsyłamy do [[...]].


\begin{definition} [funkcji silnie wypukłej]
Funkcja $f: W \rightarrow \mathbb{R}$ jest $\mu$-silnie wypukła, wtedy i tylko wtedy, gdy dla dowolnych $x,y \in W$ oraz $t \in [0,1]$ zachodzi:
\[
f(tx + (1-t)y) + \frac{\mu t(1-t)}{2}||x-y||_2^2 \leq tf(x) + (1-t)f(y) 
\]
\end{definition}

Silna wypukłość jest ważną własnością. Wiele omawianych przez nas algorytmów optymalizacji funkcji wypukłych będzie zbiegać istotnie szybciej dla funkcji silnie wypukłych. Na ćwiczeniach omówimy różne równoważne definicje i konsekwencje silnej wypukłości.


\section{Subgradient i subróżniczka}



\begin{definition}[Subgradient]
Dla danej funkcji wypukłej $f : W \rightarrow \mathbb{R}$ oraz $x \in W$ subgradientem $f$ w $x$ nazwiemy dowolny wektor $v$, taki że dla wszystkich $z \in W$:
\[
f(z) \geq f(x) + \langle v , z - x \rangle
\]
\end{definition}

\begin{definition}[Subróżniczka]
Subróżniczką funkcji wypukłej $f: W \rightarrow \mathbb{R}$ w $x \in W$ nazwiemy zbiór wszystkich subgradientów $f$ w $x$. Subróżniczkę $f$ w $x$ oznaczamy przez $\partial f(x)$. Formalnie:
\[
\partial f(x) = \{ v : \forall_{z \in W} f(z) \geq f(x) + \langle v, z - x\rangle \}
\]
\end{definition}

\begin{definition}[Epigraf]
Dla danej funkcji wypukłej $f : W \rightarrow \mathbb{R}$ epigrafem $f$ nazwiemy zbiór
\[
epi (f) = \{ (x,z) : z \geq f(x) \}
\]
\end{definition}


\begin{theorem}
Funkcja $f$ jest wypukła wtedy i tylko wtedy gdy jej epigraf jest wypukły.
\end{theorem}
\begin{proof}
Ćwiczenie.
\end{proof}

\begin{theorem}[o hiperpłaszczyźnie podpierającej]
Jeśli $f$ jest wypukła, to w każdym punkcie $x \in int(W)$ istnieje hiperpłaszczyzna podpierająca - to znaczy taki wektor $s_x$, że dla wszystkich $y \in W$ zachodzi:
\[
f(y) \geq f(x) + s_x^T (y - x)
\]
Innymi słowy, we wnętrzu dziedziny subróżniczka funkcji wypukłej jest niepusta.
\end{theorem}
\begin{proof}
Epigraf $f$ jest zbiorem wypukłym (ćwiczenia). Ponieważ $f$ jest ciągła we wnętrzu dziedziny, to Epigraf ma również niepuste wnętrze. Weźmy kulkę domkniętą $\overline{B}(x_0, \epsilon) \subset int(W)$ i niech $z = sup_{x \in \overline{B}(x_0, \epsilon)} f(x)$. Wówczas zbiór $B(x_0, \epsilon) \times (z, \infty)$ jest otwartym zbiorem zawartym w epigrafie. Stąd możemy zastosować słabe twierdzenie o oddzielaniu \ref{cor:weak_sep} dla zbiorów $epi(f)$ oraz $\{x, f(x)\}$ dla $x \in int(W)$. Istnieje zatem niezerowe $(v, \alpha) \in \mathbb{R}^n \times \mathbb{R}$, takie że:
\begin{equation}\label{eq:exist_subg}
v^T y + \alpha z \leq v^T x + \alpha f(x)
\end{equation}
dla wszystkich $y \in W$ oraz $z \geq f(y)$. Zauważmy, że $\alpha$ nie może być dodatnie. W przeciwnym wypadku dla dowolnego $y \in W$ biorąc odpowiednio duże $z$ powyższa nierówność nie byłaby spełniona. Zatem $\alpha$ jest niedodatnie. Pokażemy przez sprzeczność, że $\alpha$ musi być ujemne. Gdyby $\alpha = 0$, to mielibyśmy $v^T(y - x) \leq 0$ dla wszystkich $y \in W$, oraz dla $v \neq 0$ (bo słabe twierdzenie o oddzielaniu daje nam niezerową parę $(v, \alpha)$). Jednak ponieważ $x$ należy do wnętrza $W$, to dla dostatecznie małego dodatniego $\epsilon$ punkt $y = x + \epsilon v$ również należy do $W$. Stąd mamy $0 \geq v^T(y - x) = ||v||_2^2$, co implikuje $v = 0$ i daje sprzeczność. Zatem $\alpha$ jest liczbą ujemną, i po podzieleniu stronami przez $\alpha$ i wstawieniu $z = f(y)$ w równaniu \ref{eq:exist_subg} otrzymujemy:
\[
f(y) \geq f(x) + (-v/\alpha)^T(y - x)
\]
Stąd wektor $-(v/\alpha)$ jest subgradientem $f$ w $x$. 
\end{proof}

\begin{theorem}
Dla funkcji wypukłej $f : W \rightarrow \mathbb{R}$ punkt $x \in W$ jest minimum globalnym wtedy i tylko wtedy, gdy $0 \in \partial f(x)$.
\end{theorem}
\begin{proof}
Z definicji subgradientu, mamy że $0 \in \partial f(x)$ wtedy i tylko wtedy, gdy dla wszystkich $y \in W$ zachodzi $f(y) \geq f(x) + 0^T(y-x) = f(x)$, czyli wtw. gdy $x$ jest minimum globalnym $f$.
\end{proof}

\begin{lemma}
Dla $f: W \rightarrow \mathbb{R}$ wypukłej oraz $x \in W$, zbior $\partial f(x)$ jest wypukły i domknięty. Ponadto, dla $x \in int(W)$ jest również ograniczony, a więc zwarty.
\end{lemma}
\begin{proof}
Udowodnimy najpierw wypukłość zbioru $\partial f(x)$. Niech $s_1, s_2 \in \partial f(x)$. Wówczas dla wszystkich $y \in W$ mamy:
\[
\begin{aligned}
f(y) & \geq f(x) + s_1^T (y - x) \\
f(y) & \geq f(x) + s_2^T (y - x)
\end{aligned}
\]
Weźmy $\lambda_1, \lambda_2 \geq 0$, $\lambda_1 + \lambda_2 = 1$. Mnożąc pierwszą nierówność stronami przez $\lambda_1$, drugą przez $\lambda_2$ oraz dodając otrzymane nierówności otrzymujemy:
\[
(\lambda_1 + \lambda_2)f(y) \geq (\lambda_1 + \lambda_2) f(x) + (\lambda_1 s_1 + \lambda_2 s_2)^T (y - x)
\]
czyli:
\[
f(y) \geq f(x) + (\lambda_1 s_1 + \lambda_2 s_2)^T (y-x)
\]
dla wszystkich $y \in W$. Stąd wniosek, że $(\lambda_1 s_1 + \lambda_2 s_2) \in \partial f(x)$, a zatem zbiór $\partial f(x)$ jest wypukły.
\\
Przejdźmy do dowodu domkniętości. Niech $\{s_n\}_{n=1}^{\infty}$ będzie ciągiem elementów $\partial f(x)$ i niech $\lim_{n \rightarrow \infty} s_n = v$. Dla dowolnego $y \in W$ oraz $n \in \mathbb{N}$ mamy:
\[
f(y) \geq f(x) + s_n^T(y - x)
\]
przechodząc z $n$ do granicy, otrzymujemy $f(y) \geq f(x) + v^T(y - x)$ dla dowolnego $y \in W$. Stąd $v \in \partial f(x)$.
\\
Pozostało nam udowodnić że $\partial f(x)$ jest ograniczony, gdy $x \in int(W)$. Dla takich $x$, istnieje $\epsilon > 0$, taki że domknieta kula $B_{\epsilon} = \overline{B}(x, \epsilon)$ jest zawarta w $int(W)$. Dla dowolnego $s_x \in \partial f(x)$ zachodzi:
\[
f(y) \geq f(x) + s_x^T(y -x)
\]
dla wszystkich $ $. Stąd, biorąc stronami supremum po $y \in B_{\epsilon}$, orzymujemy:
\[
\sup_{y \in B_{\epsilon}} f(y) \geq f(x) + \sup_{y \in B_{\epsilon}} s_x^T (y - x)
\]
Ponieważ $f$ jest ciągła w zbiorze $int(W)$, to $\sup_{y \in B_{\epsilon}} f(y)$ jest skończone. Ponadto mamy 
\[
\sup_{y \in B_{\epsilon}} s_x^T (y - x) = \epsilon ||s_x||_2
\]
Stąd dla dowolnego $s_x \in \partial f(x)$ zachodzi:
\[
||s_x||_2 \leq \frac{1}{\epsilon} \left(\sup_{y \in B_{\epsilon}} f(y) - f(x) \right)
\]
zatem zbiór $\partial f(x)$ jest ograniczony.


\end{proof}

\begin{definition}
Pochodną kierunkową funkcji $f$ w kierunku $d$ nazywamy:
\[
f'(x, d) = \lim_{\lambda \rightarrow 0_+} \frac{f(x + \lambda d) - f(x)}{\lambda}
\]
\end{definition}

\begin{theorem}
Dla funkcji wypukłej $f : W \rightarrow \mathbb{R}$ oraz punktu $x \in int(W)$ pochodna kierunkowa istnieje w każdym kierunku. Ponadto:
\[
f'(x,d) = \inf_{\lambda >0} \frac{f(x + \lambda d) - f(x)}{\lambda}
\]
\end{theorem}
\begin{proof}
Wykażemy, że iloraz różnicowy jest monotoniczny, to znaczy dla $0 < t_1 < t_2$, takich że $x + t_2 d \in W$, zachodzi:
\[
\frac{f(x + t_1 d) - f(x)}{ t_1} \leq \frac{f(x + t_2 d) - f(x)}{t_2}
\]
Powyższa nierówność jest równoważna:
\[
f(x + t_1 d) \leq \frac{t_1}{t_2} f(x + t_2 d) + \frac{(t_2 - t_1)}{t_2} f(x)
\]
co wynika wprost z wypukłości funkcji $f$.
\\
Z monotoniczności ilorazu różnicowego wnioskujemy, że:
\[
\lim_{\lambda \rightarrow 0_+} \frac{f(x + \lambda d) - f(x)}{\lambda} = \inf_{\lambda > 0} \frac{f(x + \lambda d) - f(x)}{\lambda}
\]
Pozostaje jeszcze wykazać, że to infimum jest skończone. To jednak wynika z istnienia subgradientu $f$ w $x$. Jeśli bowiem $s_x \in \partial f(x)$, to mamy $f(x + \lambda d) \geq f(x) + s_x^T(\lambda d)$, stąd:
\[
\frac{f(x + \lambda d) - f(x)}{\lambda} \geq s_x^T d
\]
co kończy dowód.
\end{proof}

\begin{theorem}
Wektor $s_x$ jest subgradientem $f: W \rightarrow \mathbb{R}$ w $x \in int(W)$ wtedy i tylko wtedy, gdy $f'(x,d) \geq s_x^T d$ dla wszystkich $d \in \mathbb{R}^n$.
\end{theorem}
\begin{proof}
Wykazaliśmy w dowodzie poprzedniego twierdzenia implikację w jedną stronę. Wystarczy dowieść, że jeśli $f'(x,d) \geq s_x^T d$ dla wszystkich $d \in \mathbb{R}^n$, to $s_x$ jest subgradientem $f$ w $x$. Z monotoniczności ilorazu różnicowego dla dowolnych $d \in \mathbb{R}^n$ oraz $\lambda > 0$, takich że $x + \lambda d \in W$, mamy:
\[
\frac{f(x + \lambda d) - f(x)}{\lambda} \geq f'(x,d) \geq s_x^T d
\]
Stąd $f(x + \lambda d) \geq f(x) + s_x^T( \lambda d)$, a zatem $f(y) \geq f(x) + s_x^T(y - x)$ dla dowolnego $y \in W$.
\end{proof}

\begin{theorem} \label{thm:dir_deriv_sup}
Dla funkcji wypuklej $f: W \rightarrow \mathbb{R}$ określonej na otwartym zbiorze $W$, dowolnego $x \in W$ oraz $d \in \mathbb{R}^n \setminus \{0\}$ zachodzi:
\[
f'(x, d) = \max_{s_x \in \partial f(x)} s_x^T d 
\]
\end{theorem}
\begin{proof}
Wiemy z poprzedniego punktu, że zachodzi
\[
f'(x, d) \geq \sup_{s_x \in \partial f(x)} s_x^T d 
\]
Wykażemy nierówność w drugą stronę oraz że supremum jest przyjmowane. Niech $L = \sup \{\lambda : \lambda \geq 0, x + \lambda d \in W\}$. Określamy zbiory:
\[
\begin{aligned}
C_1 &= \left\{ (y,z) : x \in W, z > f(y) \right\} \\
C_2 &= \left\{ (y,z) : y = x + \lambda d, z = f(x) + \lambda f'(x,d), \lambda \in [0, L) \right\}
\end{aligned}
\]
Zbiór $C_1$ jest wnętrzem epigrafu $f$, jest więc wypukły. Zbiór $C_2$ jest odcinkiem, zatem również jest wypukły. Ponieważ $\lambda f'(x,d) \leq f(x + \lambda d) - f(x)$, zbiory $C_1, C_2$ są rozłączne. Z twierdzenia o oddzielaniu wnioskujemy zatem, że istnieją $a \in \mathbb{R}^n$, $b \in \mathbb{R}$, $(a,b) \neq 0$, takie że:
\begin{equation}\label{ineq:dir_deriv}
a^T y + bz \geq a^T (x + \lambda d) + b (f(x) + \lambda f'(x,d))
\end{equation}
dla $(y,z) \in C_1$ oraz ($\lambda \in [0, L)$. Skalując wektor wyznaczający hiperpłaszczyznę rozdzieljącą, możemy założyć, że $b \in \{ -1, 0, 1\}$. Podobnie jak w dowodzie istnienia subgradientów, zauważamy że nie może być $b < 0$ bo wtedy biorąc odpowiednio duże $z$ nierówność \ref{ineq:dir_deriv} nie byłaby spełniona. Nie może być również $b = 0$, bo wówczas byłoby $a \neq 0$ oraz dla wszystkich $y \in W, \lambda \in [0, L)$ mielibyśmy $a^T(y - x) \geq \lambda a^T d$, co jak łatwo zauważyć nie może mieć miejsca. Zatem $b = 1$. Biorąc w nierówności \ref{ineq:dir_deriv} granice $z \rightarrow f(y)$, $\lambda \rightarrow 0$, otrzymujemy:
\[
f(y) \geq f(x) - a^T(y - x)
\]
stąd $-a \in \partial f(x)$. Jednak biorąc w nierówności \ref{ineq:dir_deriv} $y = x$, $z \rightarrow f(x)$ oraz dowolne $\lambda > 0$, otrzymujemy:
\[
(-a)^T( \lambda d) \geq \lambda f'(x,d)
\]
stąd $(-a)^T d \geq f'(x,d)$. Co dowodzi nierówności:
\[
f'(x,d) \leq sup_{s_x \in \partial f(x)} s_x^T d 
\]
oraz że istnieje element $s_x \in \partial f(x)$ dla którego $f'(x,d) = s_x^T d$. 
\end{proof}

\begin{corollary}
Dla $f : W \rightarrow \mathbb{R}$ oraz $x \in int(W)$ funkcja $f'(x, \cdot)$ jest wypukła na $\mathbb{R}^n$ (a zatem również ciągła).
\end{corollary}
\begin{proof}
Z twierdzenia \ref{thm:dir_deriv_sup} wynika, że $f'(x,\cdot)$ jest supremum funkcji liniowych (które są wypukłe). Supremum rodziny funkcji wypukłych jest funkcją wypukłą (ćwiczenia).
\end{proof}

\begin{corollary}
Funkcja wypukła $f: W \rightarrow \mathbb{R}$ jest różniczkowalna w $x\in int(W)$ wtedy i tylko wtedy, gdy $\partial f(x)$ jest zbiorem jednoelementowym. \\ Wówczas $\partial f(x) = \{ \nabla f(x) \}$.
\end{corollary}
\begin{proof}
Wykażemy najpierw, że funkcja $f$ jest różniczkowalna w $x$ wtedy i tylko wtedy, gdy funkcja $f'(x, \cdot)$ jest liniowa. Implikacja w jedną stronę jest oczywista, gdyż wiemy z analizy, że dla funkcji różniczkowalnej zachodz $f'(x,d) = \nabla f(x)^T d$. Z drugiej storny, jeśli $f'(x, d) = a^T d$ dla pewnego $a \in \mathbb{R}^n$, 
\end{proof}



\begin{theorem}
Dla $f, g : W \rightarrow \mathbb{R}$ wypukłych, oraz $x \in int(W)$ mamy:
\[
\partial (f + g) (x) = \partial f(x) + \partial g(x)
\] 
\end{theorem}
\begin{proof}
Wykażemy najpierw zawieranie $\partial f(x) + \partial g(x) \subseteq \partial (f+g) (x)$  (które zachodzi nawet bez założenia $x \in int(W)$). Dla $s_f \in \partial f(x)$, $s_g \in \partial g(x)$ mamy dla wszystkich $y \in W$:
\[
\begin{aligned}
f(y) & \geq f(x) + s_f^T (y - x) \\
g(y) & \geq g(x) + s_g^T (y - x)
\end{aligned}
\]
Dodając obie nierówności stronami, otrzymujemy
\[
(f + g)(y) \geq (f+g)(x) + (s_f + s_g)^T(y - x)
\]
zatem $s_f + s_g \in \partial (f+g)(x)$. \\
Zawieranie w drugą stronę wykażemy przez sprzeczność. Załóżmy, że istnieje $s_x \in \partial (f+g)(x)$ który nie należy do $\partial f(x) + \partial g(x)$. Ponieważ $x \in int(W)$, to zbiory $\partial f(x), \partial g(x)$ są zwarte. Zatem suma Minkowskiego $\partial f(x) + \partial g(x)$ jest zwarta (i oczywiści wypukła). Możemy zastosować silne twierdzenie o oddzielaniu, dla zbiorów $\{s_x\}, (\partial f(x) + \partial g(x))$. Istnieją zatem $a \in \mathbb{R}^n, b \in \mathbb{R}$, taki że dla dowolnych $s_f \in \partial f(x)$, $s_g \in \partial g(x)$, zachodzi:
\[
a^T s_f + a^T s_g < b < a^T s_x 
\]
Przykładamy do powyższego wyrażenia $\sup_{s_f \in \partial f(x), s_g \in \partial g(x)}$ stronami. Korzystając z  \ref{thm:dir_deriv_sup} wnioskujemy, że:
\[
f'(x,a) + g'(x, a) \leq b < a^T s_x \leq (f+g)'(x, a)
\]
co daje nam sprzeczność, bo wprost z definicji pochodnej kierunkowej mamy $f'(x, a) + g'(x,a) = (f+g)'(x,a)$ dla dowolnego $x \in int(W)$.

\end{proof}


\section{Stożki normalne i styczne}

\begin{definition}[Stożka] Stożkiem w dowolnej przestrzeni liniowej nad $\mathbb{R}$ nazywamy dowolny zbiór $C$, taki że dla dowolnego $\lambda \geq 0$ zachodzi $x \in C \leftrightarrow \lambda x \in C$.
\end{definition}

\begin{definition}[Wektora stycznego do zbioru]
Dla danego zbioru $S$ oraz  \\ $x\in S$, powiemy że wektor $v$ jest wektorem stycznym do $S$ w $x$, wtedy i tylko wtedy gdy  istnieje ciąg punktów $x_n \in S$, $\lim_{n \rightarrow \infty} x_n = x$, oraz ciag $\lambda_n \in \mathbb{R}_+$, taki że:
\[
\lim_{n \rightarrow \infty} \frac{x_n  - x}{\lambda_n} = v 
\]
\end{definition}

\begin{theorem}
Zbiór wszystkich wektorów stycznych do zbioru $W$ w $x$ jest domknięty oraz tworzy stożek.
\end{theorem}


\begin{definition}[Stożka stycznego]
Stożkiem wektorów stycznych lub stożkiem stycznym do $W$ w $x$ nazywamy zbiór wektorów stycznych do $W$ w $x$. Często będziemy oznaczać ten zbiór przez $T_W(x)$.
\end{definition}

\begin{theorem}
Stożek styczny do zbioru wypukłego jest zbiorem wypukłym.
\end{theorem}
\begin{proof}
Ćwiczenie.
\end{proof}

\begin{definition}[Stożka dualnego] 
Stożkiem dualnym (lub polarnym) do \\ danego zbioru $S$ nazywamy zbiór:
\[
S^{\circ} = \left\{ x : \forall_{y \in S} x^T y \leq 0 \right\}
\]
\end{definition}
\begin{lemma}
Dla dowolnego $S$ stożek dualny do $S$ jest domkniętym i wypukłym stożkiem.
\end{lemma}

\begin{definition}[Stożka normalnego]
Dla zbioru $W$ oraz $x \in W$ stożkiem normalnym do $W$ w $x$ nazywamy zbiór:
\[
N_W(x) = \left\{ y : \forall_{z \in W} y^T (z-x) \leq 0 \right\}
\]
\end{definition}
\begin{lemma}
Dla dowolnego zbioru $W$ oraz $x \in W$, zbiór $N_W(x)$ jest stożkiem i jest domknięty.
\end{lemma}
\begin{proof}
Ćwiczenie.
\end{proof}

\begin{remark}
Zauważmy, że stożek normalny zbioru $W$ jest subróżniczką funkcji $f:W \rightarrow 0$ (lub indykatora wypukłego $W$).
\end{remark}

\begin{theorem} \label{thm:normal_tangent_dual}
Dla zbioru wypukłego $W$ oraz $x \in W$ mamy:
\begin{itemize}
\item $N_W(x)^{\circ} = T_W(x)$
\item $T_W(x)^{\circ} = N_W(x)$
\end{itemize}
\end{theorem}
\begin{proof}
Dowód na ćwiczeniach.
\end{proof}

\begin{theorem}
Niech $V, W$ będą zbiorami wypukłymi w $\mathbb{R}^n$, $V \subset int(W)$. Wówczas $x$ jest minimum $f:W \rightarrow \mathbb{R}$ w zbiorze $V$ wtedy i tylko wtedy gdy:
\[
0 \in \partial f(x) + N_V(x)
\]
\end{theorem}
\begin{proof}
Załóżmy, że $0 \in \partial f(x) + N_V(x)$. Wówczas istnieje wektor $\xi \in \partial f(x)$ taki że $- \xi \in N_V(x)$. Mamy:
\[
f(y) \geq f(x) + \xi(y - x)
\]
dla wszystkich $y \in W$, oraz $-\xi^T(y - x) \leq 0$ dla $y \in V$. Stąd $f(y) \geq f(x)$ dla wszyskich $y \in V$.
\\

W drugą stronę, będziemy rozumowali przez sprzeczność. Załóżmy że $x$ jest minimum $f$ na zbiorze $V$, oraz $0 \not\in \partial f(x) + N_V(x)$.
Zauważmy, że zbiór $\partial f(x) + N_V(x)$ jest domkniety i wypukły, jako suma zbiorów wypukłych, z których jeden jest zwarty a drugi domknięty. Stosując silne twierdzenie o oddzielaniu dla zbiorów $\partial f(x) + N_V(x)$ oraz $\{ 0 \}$, wnioskujemy że istnieje kierunek $d \in \mathbb{R}^n$, taki że :
\[
\sup_{y \in \partial f(x) + N_V(x)} y^T d < 0
\]
Musi zachodzić $y^T d \leq 0$ dla wszystkich $y \in N_V(x)$, a zatem $d \in T_V(x)$. Ponadto, mamy $\sup_{y \in \partial f(x)} y^T d < 0$, zatem $f'(x,d) < 0$. Wykazaliśmy więc, że $f'(x,d) < 0$ dla pewnego kierunku $d \in T_V(x)$.
Zauważmy, że dla dowolnego $y \in T_V(x)$ mamy $f'(x,\lambda (y - x)) \geq 0$ dla $y \in V, \lambda > 0$. Ponieważ $T_V(x)$ jest domknięciem zbioru $\{\lambda(y - x) : y \in V, \lambda \geq 0\}$ (dowód na ćwiczeniach) a $f'(x,d)$ funkcją wypukłą (a więc i ciągłą), to $f'(x,d) \geq 0$. Otrzymana sprzeczność dowodzi, że $0 \in \partial f(x) + N_V(x)$.
\end{proof}

\section{Rzutowanie i operator proksymalny}

\begin{definition}
Niech $W$ będzie niepustym domkniętym podzbiorem wypukłym rzeczywistej przestrzeni $\rset^n$. Funkcję:
\[
P_W(x) = argmin_{y \in W} ||x - y||
\]
\end{definition}
\begin{theorem}
Operator rzutowania jest dobrze zdefiniowany dla niepustych, wypukłych domknietych zbiorów $W$.
\end{theorem}
\begin{proof}
Niech $w \in W$ będzie dowolnym elementem. Zbiór $W \cap \overline{B}(0, ||x - w||)$ jest zwarty. Na tym zbiorze funkcja ciągła $h(\cdot) = ||x - \cdot||$ przyjmuje minimum, zatem $\inf_{y \in W} ||x - y||$ jest przyjmowane w jakimś punkcie. Pozostaje wykazać, że minimum jest przyjmowane tylko w jednym punkcie. Gdyby istniały dwa różne punkty $y_1, y_2 \in W$ w których minimum jest przyjmowane, mielibyśmy:
\[
||(y_1 + y_2)/2 - x||_2^2 < \frac{1}{2} \left( ||y_1 - x||_2^2 + ||y_2 - x||_2^2 \right)
\]
co daje sprzeczność.
\end{proof}

\begin{theorem}
Niech $W$ bedzie domkniętym wypukłym zbiorem, oraz $x$ niech będize dowolne. Wówczas $x - P_W(x) \in N_W(P_W(x))$, to znaczy dla dowolnego $y \in W$ mamy:
\[
\langle y - P_W(x), x - P_W(x) \rangle \leq 0 
\]
\end{theorem}
\begin{proof}
Dla dowolnego $y \in W$ mamy:
\[
||y - x||_2^2 \geq ||P_W(x) - x||_2^2
\]
Zatem:
\[
\langle P_W(x) - y , P_W(x) + y - 2x \rangle \leq 0
\]
Co można przepisać jako:
\[
||P_W(x) - y||_2^2 + 2 \langle P_W(x) - y , y - x \rangle \leq 0
\]
\end{proof}


Operator proksymlany można rozumieć jako pewne uogólnienie operatora rzutowania.

\begin{definition}
Dla funkcji $f : W \rightarrow \mathbb{R}$ definiujemy operator proksymalny:
\begin{equation}\label{eq:def_prox}
prox_f(x) = argmin_{y \in W} f(y) + \frac{1}{2}||x-y||_2^2 
\end{equation}
\end{definition}

\begin{theorem}
Dla zbioru wypukłego domkniętego $W \subseteq \mathbb{R}^n$, niech $g = \mathds{1}_W(\cdot)$ będzie funkcją stale równą zero na zbiorze $W$ i równą nieskończoność poza nim. Wówczas:
\[
prox_g(x) = P_W(x)
\]
dla wszystkich $x \in \mathbb{R}^n$.
\end{theorem} 

\begin{theorem}
Funkcja $prox_f$ jest dobrze określona dla funkcji wypukłych dolnie półciągłych $f : W \rightarrow \mathbb{R}$, to znaczy zbiór minimów wyrażenia po prawej stronie \ref{eq:def_prox} istnieje i jest jednoelementowy. 
\end{theorem}
\begin{proof}
Dowód na ćwiczeniach.
\end{proof}

\begin{theorem}
Dla funkcji wypukłej $f : \mathbb{R}^n \rightarrow \mathbb{R}$ zachodzi 
\[
y = prox_{\lambda f}(x) \leftrightarrow y \in x - \lambda \partial f(y)
\]
\end{theorem}
\begin{proof}
Funkcja $g(\cdot) = \lambda f(\cdot) + \frac{1}{2} ||x - \cdot||_2^2$ jest funkcją wypukłą, a jej subróżniczka w punkcie $z$ wynosi $z - x + \lambda \partial f(z)$. Punkt $y$ jest minimum globalnym $g(\cdot)$ wtw. gdy $0 \in \partial g (y)$, czyli wtw. gdy $0 \in y - x + \lambda \partial f(y)$, co jest równoważne tezie.
\end{proof}
\begin{theorem}
Dla funkcji wypukłej $f : W \rightarrow \mathbb{R}$ zachodzi 
\[
y = prox_{\lambda f}(x) \leftrightarrow y \in x - \lambda \partial f(y) - N_W(y)
\]
\end{theorem}
\begin{proof}
Tak jak poprzednio, tylko korzystamy z warunków optymalności na zbiorze ograniczonym.
\end{proof}



\section{Zadania}
\begin{problem}
Udowodnij, że zbiór $W$ jest wypukły, wtedy i tylko wtedy gdy dla dowolnych $x_1, x_2, \ldots x_n \in W$ oraz $\lambda_1, \lambda_2, \ldots \lambda_n \in \mathbb{R}_+$, $\sum_{i=1}^n \lambda_i = 1$, zachodzi
\[
\sum_{i=1}^n \lambda_i x_i
\]
\end{problem}
\begin{problem}
Niech $W \in \mathbb{R}^n$ będzie zbiorem wypukłym o niepustym wnętrzu. Wówczas:
\begin{enumerate}[a)]

\item Dla dowolnego $x \in W$ oraz $x_0 \in int(W)$ odcinek łączący $x$ i $x_0$ z pominięciem jednego z punktów początkowych należy do wnętrza $W$. Formalnie, dla $\lambda \in (0, 1]$ zachodzi
\[
\lambda x + (1 - \lambda) x_0 \in int(W)
\]

\item $W \in cl(int(W))$
\end{enumerate}
\end{problem}
\begin{problem}
Udowodnij że funkcja jest wypukła wtedy i tylko wtedy gdy jej epigraf jest wypukły.
\end{problem}

\begin{problem}
Udowodnij, że subróżniczka jest operatorem monotonicznym, to znaczy dla $s_x \in \partial f(x)$, $s_y \in \partial f(y)$ mamy:
\[
(s_x - s_y)^T (x - y) \geq 0
\]
\end{problem}

\begin{problem}
Niech $W \subseteq \mathbb{R}^n$ będzie zbiorem wypukłym o niepustym wnętrzu i niech $f : W \rightarrow \mathbb{R}$. Załóżmy że w każdym punkcie $x \in int(W)$ istnieje subgradient $f$ w $x$. Udowodnij, że funkcja $f$ jest wypukła.
\end{problem}

\begin{problem}
Niech $W \subseteq \mathbb{R}^n$ będzie zbiorem wypukłym o niepustym wnętrzu. Udowodnij, że następujące stwierdzenia są równoważne:
\begin{enumerate}[a)]
\item Funkcja $f: W \rightarrow \mathbb{R}$ jest $\mu$-silnie wypukła
\item Dla dowolnego $z \in \mathbb{R}^n$, funkcja $g(x) = f(x) - \frac{\mu}{2}||x-z||_2^2$ jest wypukła.
\item Dla dowolnego $s_x \in \partial f(x)$ oraz $y \in W$ zachodzi
\[
f(x) \geq f(y) + s_x^T(x - y) + \frac{\mu}{2} ||x-y||_2^2
\]
\item Dla dowolnych $s_x \in \partial f(x)$, $s_y \in \partial f(y)$ zachodzi
\[
(s_x - s_y)^T (x - y) \geq \mu ||x-y||_2^2
\]
\end{enumerate}
\end{problem}

\begin{problem}
Niech $f: W \rightarrow \mathbb{R}$ będzie funkcją $\mu$-silnie wypukłą. Udowodnij
\begin{enumerate}[a)]
\item Dla dowolnego $x \in W$ oraz $s_x \in \partial f(x)$ 
\[
\frac{1}{2} ||s_x||_2^2 \geq \mu (f(x) - f^*)
\]
gdzie $f^*$ jest minimum $f$ na $W$. 
\item Dla dowolnych $s_x \in \partial f(x), s_y \in \partial f(y)$ zachodzi $||s_x - s_y||_2 \geq \mu ||x-y||_2$.
\item 
\end{enumerate}

\end{problem}




\begin{problem}
Niech $W$ będzie wypukłym otwartym podzbiorem $\mathbb{R}^n$ i niech $f : W \rightarrow \mathbb{R}$ będzie klasy $C^2$
\begin{enumerate}[a)]
\item Udowodnij, że $f$ jest wypukła wtedy i tylko wtedy gdy dla każdego $x$ hessian $D^2 f(x)$ jest nieujemnie określony.
\item Udowodnij, jeśli hessian jest w każdym punkcie dodatnio określony, to $f$ jest ściśle wypukła.
\item Udowodnij, że $f$ jest $\mu$-silnie wypukła wtedy i tylko wtedy, gdy najmniejsza wartość własna hessianu w każdym punkcie jest niemniejsza niż $\mu$.
\end{enumerate}
\end{problem}




\begin{problem}
Znaleźć stożek normalny do zbioru zadanego poprzez nierówności $f_i(x) \leq 0$ dla
\begin{enumerate}[a)]
\item funkcji $f_i$ afinicznych
\item funkcji $f_i$ klasy $C_1$
\item funkcji $f_i$ wypukłych
\end{enumerate}

\end{problem}
\begin{problem}
Udowodnij, że dla zbioru wypukłego $W$ zbiór $T_W(x)$ jest domknięciem zbioru $\{ \lambda(y - x) : y \in W, \lambda \in \mathbb{R}, \lambda \geq 0\}$. 
\end{problem}
\begin{problem}
Udowodnić twierdzenie \ref{thm:normal_tangent_dual}
\end{problem}
\begin{problem}
Wykaż, że dla domknietego wypukłego stożka $S$, zachodzi $(S^{\circ})^{\circ} = S$.
\end{problem}
\begin{problem}
Udowodnij że operator rzutowania jest dobrze zdefiniowany
\end{problem}




\chapter{Metody pierwszego rzędu}
\section{Spadek gradientowy}
Spadek gradientowy jest podstawowym algorytmem optymalizacji numerycznej pierwszego rzędu funkcji klasy $C^1$.
Algorytm jest oparty na prostej intuicji - dla funkcji klasy $C^1$ gradient w punkcie $x$ jest kierunkiem najszybszego wzrostu funkcji, a minus gradient kierunkiem jej najszybszego spadku. Możemy zatem skonstruować prosty algorytm iteracyjny:
\begin{equation} \label{eq:grad_desc_def}
x_{k+1} = x_k - \steps_{k+1} \nabla f(x_k)
\end{equation}
dla pewnego $x_0$ oraz ciągu $\{\steps_{k} \}_{k=1}^{\infty}$.
Powyższa iteracja może być rozumiana jako dyskretyzacja równania różniczkowego zwyczajnego:
\[
\dot{x}(t) = - \nabla f(x(t))
\]
Rozwiązanie powyższego równania (przy warunku początkowym $x(0) = x_0$) nazywamy potokiem gradientowym $f$ (o początku w $x_0$).
\\
Dla funkcji wypukłych istnieje dokładna teoretyczna analiza zbieżności algorytmu zadanego iteracją \ref{eq:grad_desc_def}, którą przedstawimy w tym podrozdziale. Zaczniemy jednak od analizy zbieżności potoku gradientowego $f$ do minimum $f$, w przypadku gdy funkcja $f$ jest wypukła. Jak się później okaże, analiza zbieżności spadku gradientowgo jest "dyskretną" wersją analizy zbieżności potoku gradientowego.
\begin{lemma}
Niech $W$ będzie otwartym i wpypukłym podzbiorem $\rset^n$, i niech $f : W \rightarrow \rset$ będzie funkcją $m$-silnie wypukłą klasy $C^1$. Niech $x_0 \in W$, i niech $x(t)$ będzie potokiem gradientowym o początku w $x_0$. Wówczas dla dowolnego $t > 0$ oraz $z \in W$ mamy:
\[
f(x(t)) - f(z) + \frac{m}{2}||x(t) - z||_2^2 \leq - \frac{d}{dt} \frac{1}{2} ||x(t) - z||_2^2
\] 
\end{lemma}
\begin{proof}
\[
\begin{aligned}
\frac{d}{dt} \left( \frac{1}{2}||x(t) - z||_2^2 \right) & = \langle \dot{x}(t), x(t) - z \rangle \\
& = \langle \nabla f(x(t)), z - x(t) \rangle \\
& \leq f(z) - f(x(t)) - \frac{m}{2}||x(t) - z||_2^2
\end{aligned}
\]
\end{proof}
\begin{lemma}
Niech $W$ będzie otwartym i wpypukłym podzbiorem $\rset^n$, i niech $f : W \rightarrow \rset$ będzie funkcją wypukłą klasy $C^1$. Niech $x_0 \in W$, i niech $x(t)$ będzie potokiem gradientowym o początku w $x_0$. Zakładają,c że w $W$ istnieje minimum $x_*$ fukcji $f$, mamy:
\[
f(x(t)) - f(x_*) \leq \frac{1}{2t}||x_0 - x_*||_2^2
\]
\end{lemma}
\begin{proof}
Korzystając z poprzedniego lematu (biorąc $m=0$) i całkując od $0$ do $t$ mamy:
\[
2 \left( \int_0^t f(x(s)) - f(x_*) ds \right) \leq ||x_0 - x_*||_2^2 - ||x_t - x_*||_2^2  
\]
następnie korzystając z nierówności $f(x(s_1)) \geq f(x(s_2))$ dla $s_2 \geq s_1$ (co wynika z $d/dt f(x(t)) = - ||\nabla f(x(t))||_2^2 \leq 0$), wnioskujemy:
\[
2t \left( f(x(t) - f(x_*) \right) \leq ||x_0 - x_*||_2^2 - ||x_t - x_*||_2^2  
\]
z powyższego teza wynika bezpośrednio przez opuszczenie niedodatniego czynnika po prawej stronie i podzielenie przez $2t$ stronami.
\end{proof}
\begin{lemma}
Niech $W$ będzie otwartym i wpypukłym podzbiorem $\rset^n$, i niech $f : W \rightarrow \rset$ będzie funkcją $m$-silnie wypukłą klasy $C^1$. Niech $x_0 \in W$, i niech $x(t)$ będzie potokiem gradientowym o początku w $x_0$. Zakładają,c że w $W$ istnieje minimum $x_*$ fukcji $f$, mamy:
\[
||x(t) - x_*||_2^2 \leq e^{-mt} ||x_0 - x_*||_2^2
\]
\end{lemma}
\begin{proof}
Ponieważ $f(x(t)) \geq f(x_*)$, mamy:
\[
\frac{d}{dt} ||x(t) - x_*||_2^2 \leq - m||x(t) - x_*||_2^2
\]
Korzystając z nierówności Gronwalla, otrzymujemy tezę.
\end{proof}

Dowód dla dyskretnej iteracji zaczniemy od dowodu następującego lematu:
\begin{lemma}
Niech $W$ bedzie wypukłym otwartym podzbiorem $\rset^n$. Dla funkcji $f : W \rightarrow \rset$ klasy $C^1$ (niekoniecznie wypukłej) o $L$-Lipshitzowskim gradiencie zachodzi:
\[
f(z) \leq f(x) + \nabla f(x)^T (y - x) + \frac{L}{2}||y - x||_2^2
\]
\end{lemma}
\begin{proof}
...
\end{proof}
\begin{corollary}
Dla $\steps_{k+1} \leq \frac{1}{L}$ mamy $f(x_{k+1}) \leq f(x_k)$.
\end{corollary}

\begin{theorem}
\[
2\steps_{k+1} \left(f(x_{k+1} - f(z) \right) \leq ||x_k - z||_2^2 + ||x_{k+1} -z||_2^2 + (1 - \steps_{k+1}/L) ||\nabla f(x_k)||_2^2 
\]
\end{theorem}
\begin{theorem}
Analiza zbieżności dla $\steps_k \leq L^{-1}$.
\end{theorem}
\begin{theorem}
Analiza zbieżności dla nierówności KŁ / QG.
\end{theorem}
\begin{theorem}
Rzutowany spadek gradientowy
\end{theorem}
\section{Iteracja proksymalna}

\begin{theorem}
Niech $x_{k+1} = prox_{\steps_{k+1} f} (x_k)$. Wówczas dla dowolnego $z$:
\[
2\steps_{k+1}(f(x_{k+1}) - f(z)) \leq ||x_k - z||_2^2 - ||x_{k+1} - z||_2^2 - ||x_{k} - x_{k+1}||_2^2
\]
\end{theorem}

\section{Proksymalny spadek gradientowy}
algorytm
lemat
analiza
\section{Metoda subgradientowa}
algorytm
analiza

\section{Mirror descent}

Sebastian Bubeck, "Five miracles of mirror descent" na YT. Dziewięć wykładów o długości 1h każdy. Nie do końca o optymalizacji, ale dużo ciekawych informacji o MD.

\section{Metody stochastyczne}

\chapter{Przyśpieszenie Nesterova}

\section{Algorytm i analiza}

\section{Restarty}


\chapter{Teoria dualności}
\section{Sformułowanie}
Omówimy teorię dualności dla ogólnych problemów optymalizacyjnych:
\[
\begin{array}{c}
min f(x) \\
g_i(x) \leq 0 \\
h_i(x) = 0
\end{array}
\]
Lagrangian \\
Punkt siodłowy \\
Postać dla programów liniowych
\section{Przykłady użycia}


\chapter{Metody pierwotno-dualne}

\chapter{ADMM}


\end{document}