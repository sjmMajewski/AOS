\documentclass[10pt,a4paper,draft]{report}
\usepackage{polski}
\usepackage[utf8]{inputenc}
\usepackage{amsmath}
\usepackage{amsfonts}
\usepackage{amssymb}
\usepackage{dsfont}
\usepackage{enumerate}
\usepackage{aliascnt}
\usepackage{hyperref}
\usepackage{cleveref}


\begin{document}

\newcommand{\rset}{\mathbb{R}}
\newcommand{\steps}{\gamma}

\newtheorem{theorem}{Twierdzenie}
\crefname{theorem}{theorem}{Theorems}
\Crefname{Theorem}{Theorem}{Theorems}


\newaliascnt{lemma}{theorem}
\newtheorem{lemma}[lemma]{Lemma}
\aliascntresetthe{lemma}
\crefname{lemma}{lemma}{lemmas}
\Crefname{Lemma}{Lemma}{Lemmas}

\newaliascnt{corollary}{theorem}
\newtheorem{corollary}[corollary]{Wniosek}
\aliascntresetthe{corollary}
\crefname{corollary}{corollary}{corollaries}
\Crefname{Corollary}{Corollary}{Corollaries}

\newaliascnt{proposition}{theorem}
\newtheorem{proposition}[proposition]{Proposition}
\aliascntresetthe{proposition}
\crefname{proposition}{proposition}{propositions}
\Crefname{Proposition}{Proposition}{Propositions}

\newaliascnt{definition}{theorem}
\newtheorem{definition}[definition]{Definicja}
\aliascntresetthe{definition}
\crefname{definition}{definition}{definitions}
\Crefname{Definition}{Definition}{Definitions}

\newaliascnt{remark}{theorem}
\newtheorem{remark}[remark]{Remark}
\aliascntresetthe{remark}
\crefname{remark}{remark}{remarks}
\Crefname{Remark}{Remark}{Remarks}


\newtheorem{example}[theorem]{Example}
\crefname{example}{example}{examples}
\Crefname{Example}{Example}{Examples}

\newtheorem{problem}{Zadanie}
\crefname{problem}{problem}{problems}
\Crefname{Problem}{Problem}{Problems}

\newenvironment{proof}{\textbf{Dowód}}

\title{Algorytmy optymalizacji w statystyce \\ Zadania domowe}

\begin{problem}
Niech $W$ będzie zbiorem wypukłym i $f : W \rightarrow \rset$. Udowodnij że $f$ jest $m$-silnie wypukła, wtedy i tylko wtedy, gdy funkcja:
\[
g(x) = f(x) - \frac{m}{2}||x||_2^2
\]
jest wypukła.
\end{problem}
\begin{problem}
Udowodnij, że dla domkniętego wypukłego stożka $S$ \\ zachodzi $S = \left( S^{\circ}\right)^{\circ}$.
\end{problem}

\begin{problem}
Niech $W$ będzie zbiorem wypkłym domkniętym, a $f: W \rightarrow \rset$ niech będzie funkcją wypukłą $L$-gładką. Pojedyncza iteracja rzutowanego spadku gradientowego wygląda następująco:
\[
x_{k+1} = P_W(x_k - \steps_{k+1} \nabla f(x_{k+1})
\]
Zakładając, że $0 \leq \steps_{k+1} \leq L^{-1}$, oraz że $f$ ma na zbiorze $W$ minimumw w $x_*$, udowodnić nierówność: 
\[
2\steps_{k+1} \left( f(x_{k+1} - f(x_*) \right) \leq ||x_{k} - x_*||_2^2 - ||x_{k+1} - x_*||_2^2 
\]
\textit{Wskazówka}: Operator rzutowania można traktować jako operator proksymalny, wówczas rzutowany spadek gradientowy można traktować jako pewną wersję proksymalnego spadku gradientowego. Jest to nawet sugerowane, bo celem tego zadania jest żeby wszyscy sobie chociaż raz przepracowali dowód proksymalnego gradientu. Dowód ten można znaleźć na przykład w pracy "A Fast Iterative Shrinkage-Thresholding Algorithm for Linear Inverse Problems", powinien też pojawić się niedługo w skrypcie. Może się przydać dowodzona na ćwiczeniach zależność $x - P_W(x) \in N_W(P_W(x))$.
\end{problem}


\end{document}
